\documentclass{article}
\usepackage{graphicx} % Required for inserting images
\usepackage{longtable} % For multi-page tables

\title{dtu-machine-learning report}
\author{rcdane }
\date{February 2024}

\begin{document}

\maketitle

\section{Introduction}

\section{Notes}
dataset:
https://www.kaggle.com/datasets/kumarajarshi/life-expectancy-who/data

regression:
Life Expectancy

classification:
Developed/Developing (binary)


quick answers to questions:
\begin{enumerate}
    \item Explain what your data is about. I.e. what is the overall problem of
interest?\\
    The dataset is from the WHO. The focus is on how different attributes affect life expectancy. There is a focus on immunization.\\
    \item Provide a reference to where you obtained the data.\\ 
    https://www.kaggle.com/datasets/kumarajarshi/life-expectancy-who/data \\
    Need to find better link \\
    \item Summarize previous analysis of the data.\\
    Insert summary\\
    \item
\end{enumerate}

% THIS IS THE START OF ANDRIS' SECTION -------

\newlength{\rowsep}
\setlength{\rowsep}{8pt} % For separating rows in the table

\pagebreak
\section{The attributes of the data}

The dataset has 22 properties, which are listed below in Table 1.

\begin{longtable}[htbp]{p{8cm}cc}
  \textbf{\textit{Name and description}} & \textbf{\textit{Numerical Type}} & \textbf{\textit{Type}} \\[8pt]
  \hline
  \endfirsthead
  
  \multicolumn{3}{c}%
  {{\tablename\ \thetable{} -- continued from previous page}} \\
  \hline
  \textbf{\textit{Name and description}} & \textbf{\textit{Numerical Type}} & \textbf{\textit{Type}} \\[8pt]
  \hline
  \endhead
  
  \hline \multicolumn{3}{r}{{Continued on next page}} \\ \hline
  \endfoot
  
  \hline
  \endlastfoot

  \textbf{Country} & Discrete & Nominal \\ Country in which the measurement took place. \\[\rowsep]
  \textbf{Year} & Discrete & Interval \\ Year in which the measurement took place. \\[\rowsep]
  \textbf{Status} & Discrete & Nominal \\ Indicating if the country is Developed or Developing. \\[\rowsep]
  \textbf{Life expectancy}  & Continuous & Ratio \\ Given as age in years. \\[\rowsep]
  \textbf{Adult Mortality} & Discrete & Ratio \\ Number of adult deaths between 15 and 60 years per 1000 population, for both sexes. \\[\rowsep]
  \textbf{Infant deaths} & Discrete & Ratio \\ Number of infant deaths per 1000 population. \\[\rowsep]
  \textbf{Alcohol} & Continuous & Ratio \\ Alcohol consumption per capita(15+), in litres of pure alcohol. \\[\rowsep]
  \textbf{Percentage expenditure} & Continuous & Ratio \\ Expenditure on health as a percentage of Gross Domestic Product per capita (\%). \\[\rowsep]
  \textbf{Hepatitis} B & Discrete & Ratio \\ HepB immunization coverage among 1-year-olds (\%). \\[\rowsep]
  \textbf{Measles} & Discrete & Ratio \\ Number of reported cases per 1000 population. \\[\rowsep]
  \textbf{BMI} & Continuous & Interval \\ Average Body Mass Index of entire population. \\[\rowsep]
  \textbf{Under-five deaths} & Discrete & Ratio \\ Number of under-five deaths per 1000 population. \\[\rowsep]
  \textbf{Polio} & Discrete & Ratio \\ Polio (Pol3) immunization coverage among 1-year-olds (\%). \\[\rowsep]
  \textbf{Total expenditure} & Continuous & Ratio \\ General government expenditure on health as a percentage of total government expenditure (\%). \\[\rowsep]
  \textbf{Diphtheria} & Discrete & Ratio \\ Diphtheria tetanus toxoid and pertussis (DTP3) immunization coverage among 1-year-olds (\%). \\[\rowsep]
  \textbf{HIV/AIDS} & Continuous & Ratio \\ Deaths per 1000 live births HIV/AIDS (0-4 years). \\[\rowsep]
  \textbf{GDP} & Continuous & Ratio \\ Gross Domestic Product per capita in USD. \\[\rowsep]
  \textbf{Population} & Discrete & Ratio \\ Population of the country. \\[\rowsep]
  \textbf{Thinness 10 - 19} & Continuous & Ratio \\ Prevalence of thinness among children and adolescents from Age 10 to 19 (\%). \\[\rowsep]
  \textbf{Thinness 5 - 9} & Continuous & Ratio \\ Prevalence of thinness among children from Age 5 to 9 (\%). \\[\rowsep]
  \textbf{Income comp. of resources} & Continuous & Ratio \\ Human Development Index in terms of income composition of resources (index ranging from 0 to 1). \\[\rowsep]
  \textbf{Schooling} & Continuous & Ratio \\ Number of years of Schooling. \\[\rowsep]
  \caption{The properties of the dataset} \label{tab:my_table} \\
\end{longtable}

% THIS IS THE END OF ANDRIS' SECTION -------

\end{document}
